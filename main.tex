%! Author = andot
%! Date = 11/23/2025

% Preamble
\documentclass[11pt, a4paper]{article}

% Packages
\usepackage{amsmath}
\usepackage{ragged2e}
\usepackage{geometry}
\usepackage{graphicx}
\usepackage{hyperref}
\usepackage{array}
\usepackage{makecell}
\usepackage{float}
\usepackage{multicol}
\usepackage{titlesec}
\usepackage{tocloft} % for TOC formatting

% Define a new section level: subsubsubsection
\titleclass{\subsubsubsection}{straight}[\subsubsection]

\newcounter{subsubsubsection}[subsubsection]
\renewcommand\thesubsubsubsection{\thesubsubsection.\arabic{subsubsubsection}}

\titleformat{\subsubsubsection}
    {\normalfont\normalsize\bfseries}{\thesubsubsubsection}{1em}{}

\titlespacing*{\subsubsubsection}
    {0pt}{3.25ex plus 1ex minus .2ex}{1.5ex plus .2ex}

% Ensure it appears in ToC
\setcounter{secnumdepth}{4}
\setcounter{tocdepth}{4}

% Force subsubsubsection TOC entries to be line by line
\makeatletter
\newcommand{\l@subsubsubsection}[2]{
    \ifnum \c@tocdepth >3
        \vspace{-14pt}
        \indent \@dottedtocline{4}{7em}{4em}{#1}{#2}
    \fi}
\makeatother


% Document
\title{Connection guide for Raspberry Pi and Arduino}
\author{Trieu An Do}
\begin{document}
    \maketitle\centering{BETTER FINISH IT BEFORE CHRISTMAS}

    \justifying
    \tableofcontents

    \newpage
    \newgeometry{margin = 2cm}
    \section{Motor}
        % subsection{Product link}
        % DIT CON ME MAY CHUA CO LINK
        \subsection{Material}
            \raggedright
            Motor: 9V, 24V, speed changing by PWM\\
            Raspberry Pi: Pi 4 Model B, Pi 5/Arduino: Uno Rev 3\\
            GPIO cables\\
            Motor driver L298N\\
            L298N Pinout:

            \begin{figure}[H]                                      
                \centering
                \includegraphics[width=0.35\textwidth]{image/L298Npinout.jpg}
                \caption{L298N Pin Layout}
                \label{fig: l298npin}
            \end{figure}
            1 L298N can controll maximum 2 independent motors.\\
            Input1 controls OUT1 and so on.\\
            Control by H-bridge:
            \begin{center}
                \small
                \begin{tabular}{|p{2cm}|p{2cm}|p{5cm}|}
                \hline
                \vspace{-5pt}\makecell{Low} & \vspace{-5pt}\makecell{Low} & \vspace{-5pt}\makecell{Stop}\\[4pt] \hline
                \vspace{-5pt}\makecell{High} & \vspace{-5pt}\makecell{Low} & \vspace{-5pt}\makecell{Clockwise/Anti-Clockwise}\\[4pt] \hline
                \vspace{-5pt}\makecell{High} & \vspace{-5pt}\makecell{Low} & \vspace{-5pt}\makecell{Clockwise/Anti-Clockwise}\\[4pt] \hline
                \vspace{-5pt}\makecell{High} & \vspace{-5pt}\makecell{High} & \vspace{-5pt}\makecell{Brake}\\[4pt] \hline
                \end{tabular}
            \end{center}

        \subsection{Connection with Raspberry Pi}
            \subsubsection{Connection scheme}
                The connection scheme to connect motor to rbp using l298n
                \begin{figure}[H]
                    \centering
                    \includegraphics[width=0.5\textwidth]{image/motorscheme.jpg}
                    \caption{Motor connection scheme}
                    \label{fig: motorrbprscheme}
                \end{figure}

            \subsubsection{Actual wiring}
                The image of real connection
                \begin{figure}[H]
                    \centering
                    \includegraphics[width=0.5\textwidth]{image/motorrealimage.jpg}
                    \caption{Real connection image}
                    \label{fig: motorrbpreal}
                \end{figure}

            \subsubsection{Code example}
                \begin{multicols}{2}
                \begin{verbatim}
    import RPi.GPIO as GPIO
    import time

    # Pin setup
    IN1 = 17
    IN2 = 27
    ENA = 22  # PWM pin

    GPIO.setmode(GPIO.BCM)
    GPIO.setup(IN1, GPIO.OUT)
    GPIO.setup(IN2, GPIO.OUT)
    GPIO.setup(ENA, GPIO.OUT)

    # Setup PWM at 1000 Hz
    pwm = GPIO.PWM(ENA, 1000)
    pwm.start(0)  # start with speed 0

    def motor_forward(speed):
        GPIO.output(IN1, GPIO.HIGH)
        GPIO.output(IN2, GPIO.LOW)
        pwm.ChangeDutyCycle(speed)

    def motor_backward(speed):
        GPIO.output(IN1, GPIO.LOW)
        GPIO.output(IN2, GPIO.HIGH)
        pwm.ChangeDutyCycle(speed)

    def motor_stop():
        GPIO.output(IN1, GPIO.LOW)
        GPIO.output(IN2, GPIO.LOW)
        pwm.ChangeDutyCycle(0)

    try:
        print("Motor forward")
        motor_forward(60)   # 60% speed
        time.sleep(2)

        print("Motor backward")
        motor_backward(60)  # 60% speed
        time.sleep(2)

        print("Stop")
        motor_stop()
        time.sleep(1)

    except KeyboardInterrupt:
        pass

    finally:
        pwm.stop()
        GPIO.cleanup()
                \end{verbatim}
                \end{multicols}

            \subsubsection{Remark for connection with RBP}
                \begin{itemize}
                    \item 1 L298N occupies at least 4 GPIOs on RBP (for 2 motors).
                    \vspace{-5pt}\item L298N can be powerred by RBP or external power source.
                \end{itemize}
        
        \subsection{Connection with Arduino}
            \subsubsection{Connection scheme}
                The connection scheme to connect motors to arduino uno rev 3 using l298n is the same as connection with RBP

            \subsubsection{Actual wiring}
                The image of real connection
                \begin{figure}[H]
                    \centering
                    %\includegraphics[width=0.5\textwidth]{image/motorrealimage.jpg}
                    \caption{Real connection image}
                    %\label{fig: motorrbpreal}
                \end{figure}

            \subsubsection{Code example}
                \begin{multicols}{2}
                \begin{verbatim}
    // Motor control pins
    int IN1 = 7;
    int IN2 = 8;
    int ENA = 9;   // must be a PWM pin (~)

    void setup() {
    pinMode(IN1, OUTPUT);
    pinMode(IN2, OUTPUT);
    pinMode(ENA, OUTPUT);

    // Start with motor stopped
    analogWrite(ENA, 0);
    }

    void motorForward(int speed) {
    digitalWrite(IN1, HIGH);
    digitalWrite(IN2, LOW);
    analogWrite(ENA, speed);   // 0-255
    }

    void motorBackward(int speed) {
    digitalWrite(IN1, LOW);
    digitalWrite(IN2, HIGH);
    analogWrite(ENA, speed);   // 0-255
    }

    void motorStop() {
    digitalWrite(IN1, LOW);
    digitalWrite(IN2, LOW);
    analogWrite(ENA, 0);
    }

    void loop() {
    // Forward for 2 seconds
    motorForward(150);    // about 60% speed
    delay(2000);

    // Backward for 2 seconds
    motorBackward(150);
    delay(2000);

    // Stop for 1 second
    motorStop();
    delay(1000);
    }
                \end{verbatim}
                \end{multicols}

            \subsubsection{Remark for conection with Arduino}
                \begin{itemize}
                    \item 1 L298N occupies at least 4 GPIOs on Arduino (for 2 motors).
                    \vspace{-5pt}\item Recommend connection: Motor $\rightarrow$ L298N $\rightarrow$ Arduino $\rightarrow$ RBP
                \end{itemize}

    \section{NFC Reader}
        \subsection{Material}
            NFC Readers: PN532 V3\\
            NFC Tag\\
            Raspberry Pi: Pi 4 Model B, Pi 5(error)/Arduino: Uno Rev 3\\
            GPIO cables\\
            I2C Multiplexer(for more than 2 readers)\\

        \subsection{Connection with RBP}
            If all the readers have different I2C addresses, then we only need one bus, but if they have same address, and that address can't be changed, then here come some potential solutions:\\
            \subsubsection{2-reader connection: 1 I2C and 1 SPI}
                \subsubsubsection{Conection scheme}
                    \begin{figure}[H]                                      
                        \centering
                        \includegraphics[width=0.7\textwidth]{image/pn532scheme.png}
                        \caption{NFC Reader connection scheme}
                        \label{fig: l298npin}
                    \end{figure}

                \subsubsubsection{Actual wiring}
                    \begin{figure}[H]
                        \centering
                        \begin{minipage}{0.45\textwidth}
                            \centering
                            \includegraphics[width=\linewidth]{image/pn532real1.jpg}
                            \caption{Overview}
                        \end{minipage}
                        \hfill
                        \begin{minipage}{0.45\textwidth}
                            \centering
                            \includegraphics[width=\linewidth]{image/pn532real2.jpg}
                            \caption{Pin connection zoom in}
                        \end{minipage}
                    \end{figure}

                \subsubsubsection{Remark for connection with RBP}
                    \begin{itemize}
                        \item For 2-reader connection, 8 GPIO pins are used (including the VCC and Ground)
                    \end{itemize}

            \subsubsection{More-than-2-reader connection: I2C multiplexer}
                \textbf{Create new buses:}
                \begin{itemize}
                    \item \url{https://www.instructables.com/Raspberry-PI-Multiple-I2c-Devices/}\\
                \end{itemize}
            
                \textbf{I2C Multiplexer (max 8 readers):}
                \begin{itemize}
                    \item \url{https://www.adafruit.com/product/2717?srsltid=AfmBOorhzn_5ry0sL1WOVAlTeGnU4hh1WHF7dkA4JKQbefiaHu6vNRcs}\\
                    \item \url{https://learn.adafruit.com/adafruit-tca9548a-1-to-8-i2c-multiplexer-breakout}\\
                \end{itemize}

        \subsection{Connection with Arduino}
            \begin{itemize}
                \item https://forum.arduino.cc/t/4-pn532-spi-arduino-nano/1041037
            \end{itemize}

    \section{Sensor}
        \subsection{Material}
            Sensor: Fischertechnik Photo transistor\\
            I2C GPIO Expander: Adafruit MCP23017\\
            Raspberry Pi: Pi 4 Model B, Pi 5(error)/Arduino: Uno Rev 3\\
            GPIO cables\\

        \subsection{Connection with RBP}
            \subsubsection{Conection scheme}
                \begin{figure}[H]                                      
                    \centering
                    \includegraphics[width=0.7\textwidth]{image/sensorscheme.jpg}
                    \caption{Sensor connection scheme}
                \end{figure}

            \subsubsection{Actual wiring}
                \begin{figure}[H]                                      
                    \centering
                    \includegraphics[width=0.5\textwidth]{image/sensorreal1.jpg}
                    \caption{Overview}
                \end{figure}

                \begin{figure}[H]
                    \centering
                    \begin{minipage}{0.3\textwidth}
                        \centering
                        \includegraphics[width=\linewidth]{image/sensorreal2.jpg}
                        \caption{Zoom in 1}
                    \end{minipage}\hfill
                    \begin{minipage}{0.3\textwidth}
                        \centering
                        \includegraphics[width=\linewidth]{image/sensorreal3.jpg}
                        \caption{Zoom in 2}
                    \end{minipage}\hfill
                    \begin{minipage}{0.3\textwidth}
                        \centering
                        \includegraphics[width=\linewidth]{image/sensorreal4.jpg}
                        \caption{Zoom in 3}
                    \end{minipage}
                \end{figure}

            \subsubsection{Remark for connection with RBP}
                    \begin{itemize}
                        \item 1 MCP23017 can connect to 16 sensorscheme
                        \item The MCP23017 uses the I2C bus on RBP, but the I2C address can be set
                        \item 3 address pins $\rightarrow$ 1 I2C bus on RBP can theoretically connect to 8 MCP23017
                        \item 1 RBP can connect to 2 PN532 and multiple MCP23017
                    \end{itemize}

        \subsection{Connection with Arduino}
    
    \section{Compressor}



\end{document}